\section{Policy Gradient Methods}

The goal of this problem is to experiment with policy gradient and its variants, including variance reduction methods. Your goal will be to set up policy gradient for both continuous and discrete environments, and implement a neural network baseline for variance reduction. The script for running the policy gradient algorithm is setup in \texttt{run.py}, and everything that you need to implement is in the files \texttt{baseline\_network.py}, \texttt{mlp.py}, \texttt{policy.py} and \texttt{policy\_gradient.py} within the submission folder. Each submission script has detailed instructions for each implementation task. We have also provided an overview of key steps in the algorithm below: \\


\textbf{REINFORCE}

Recall the policy gradient theorem:
\[ \nabla_\theta J(\theta) = \mathbb E_{\pi_\theta} \left[ \nabla_\theta \log\pi_\theta(a \mid s) Q^{\pi_\theta} (s,a) \right] \]
REINFORCE is a Monte Carlo policy gradient algorithm, so we will be using the sampled returns $G_t$ as unbiased estimates of $Q^{\pi_\theta}(s,a)$. 
The REINFORCE estimator can be expressed as the gradient of the following objective function:
\[ J(\theta) = \frac{1}{\sum T_i} \sum_{i=1}^{\mid D \mid} \sum_{t=1}^{T_i} \log(\pi_\theta(a^i_t \mid s^i_t)) G^i_t \]
where $D$ is the set of all trajectories collected by policy $\pi_\theta$, and $\tau^i =(s^i_0, a^i_0, r^i_0, s^i_1, \dots, s^i_{T_i}, a^i_{T_i}, r^i_{T_i})$ is trajectory $i$. \\

\textbf{Baseline}

One difficulty of training with the REINFORCE algorithm is that the Monte Carlo sampled return(s) $G_t$ can have high variance. To reduce variance, we subtract a baseline $b_{\phi}(s)$ from the estimated returns when computing the policy gradient. A good baseline is the state value function, $V^{\pi_\theta}(s)$, which requires a training update to $\phi$ to minimize the following mean-squared error loss:
\[ L_{\text{MSE}}(\phi) = \frac{1}{\sum T_i} \sum_{i=1}^{\mid D \mid} \sum_{t=1}^{T_i} (G^i_t - b_{\phi}(s^i_t))^2\] \\

\textbf{Advantage Normalization}

After subtracting the baseline, we get the following new objective function:

\[ J(\theta) = \frac{1}{\sum T_i} \sum_{i=1}^{\mid D \mid} \sum_{t=1}^{T_i} \log(\pi_\theta(a^i_t \mid s^i_t)) \hat{A}^i_t \]

where

\[\hat{A}^i_t = G^i_t - b_{\phi}(s^i_t)\]

A second variance reduction technique is to normalize the computed advantages, $\hat{A}^i_t$, so that they have mean $0$ and standard deviation $1$. From a theoretical perspective, we can consider centering the advantages to be simply adjusting the advantages by a constant baseline, which does not change the policy gradient. Likewise, rescaling the advantages effectively changes the learning rate by a factor of $1/\sigma$, where $\sigma$ is the standard deviation of the empirical advantages. \\

\textit{Note: for the following coding questions some scripts contain member functions of different classes with the same name. In order to distinguish which function we refer to in each question we use the following syntax ~class::function~.}
\clearpage


\begin{enumerate}[(a)]

	\item \points{1a}

Implement ~build_mlp~ within ~submission/mlp.py~ which will be used to construct a multi layer perceptron based on input argument values.

% Note: The "batch size" for all the arguments is $\sum T_i$ since we already flattened out all the episode observations, actions, and rewards for you.


% We have provided some basic tests to sanity check your implementation. \textbf{Please note that the tests are not comprehensive, and passing them does not guarantee a correct implementation}. Use the following command to run the tests:
% \begin{verbatim}
% python run_basic_tests.py
% \end{verbatim}
% You can also add additional tests of your own design in \texttt{tests/test\_basic.py}.

	\item \points{1b}

Implement the following functions within ~submission/baseline_network.py~ to create the baseline network for our policy gradient implementation: 

\begin{itemize}
	\item ~BaselineNetwork::__init__~ 
	\item ~BaselineNetwork::forward~ 
	\item ~BaselineNetwork::calculate_advantage~ 
	\item ~BaselineNetwork::update_baseline~
\end{itemize}

	\item \points{1c}

Implement ~PolicyGradient::init_policy~ within ~policy_gradient.py~ in order to initialize a network and optimizer for our implementation of policy gradient.

	\item \points{1d}

Implement ~PolicyGradient::get_returns~ within ~policy_gradient.py~ in order to calculate returns $G_{t}$ given data about specific trajectories.

	\item \points{1e}

In this question, we will define the conditional probability distribution over actions for both discrete and continuous environments. Implement the following functions in ~submission/policy.py~ in order to define these distributions and how we sample from them: 

\begin{itemize}
	\item ~BasePolicy::act~ 
	\item ~CategoricalPolicy::action_distribution~ 
	\item ~GaussianPolicy::__init__~ 
	\item ~GaussianPolicy::std~
	\item ~GaussianPolicy::action_distribution~ 
\end{itemize}

	\item \points{1f}

Implement the following functions in ~submission/policy_gradient.py~: 

\begin{itemize}
	\item ~PolicyGradient::update_policy~ 
	\item ~PolicyGradient::normalize_advantage~ 
\end{itemize}

This will complete the ~PolicyGradient~ class by providing a network update rule as well as the option to normalize advantages we have calculated.

	\item \points{1g}

In this question, you will train your policy gradient implementation on three different environments.

\begin{itemize}
	\item \href{https://www.gymlibrary.dev/environments/classic_control/cart_pole/}{CartPole-v1}
	\item \href{https://www.gymlibrary.dev/environments/mujoco/inverted_pendulum/}{InvertedPendulum-v4}
	\item \href{https://www.gymlibrary.dev/environments/mujoco/half_cheetah/}{HalfCheetah-v4}
\end{itemize}

For each of the 3 environments, choose 3 random seeds and run your policy gradient implementation both with and without a baseline.

\clearpage
\textbf{Training Policy Gradient}

The general form for running your policy gradient implementation is as follows (where config\_filename is the name of a yaml file in the ~src/config~ folder with your training configuration):

\begin{lstlisting}
$ python run.py --config_filename config_filename
\end{lstlisting}

Depending on the configuration file details you provide this command may train models for more than one seed which can take a while (especially for \textit{HalfCheetah-v4}). As a result, you may wish to run these jobs as background processes in which case you may run the command as follows:

\begin{lstlisting}
$ nohup python run.py --config_filename config_filename &
\end{lstlisting}

This will prevent the python process from ending when you close your terminal window as well as running the command in the background. Additionally, in the directory where you run the command a ~nohup.out~ file is created and contains the standard output from the process that you are running. Feel free to make use of this form of the command for the longer running processes.

\textbf{Altering the Training Configuration}

We have provided you with sample configuration files in the assignment ~config~ folder for you to use such as ~config/cartpole_baseline.yml~. These configurations will be suitable to use directly in training your agent without altering the config file.

However, you may optionally alter the training configuration files directly to run your policy gradient implementation with different settings. Below we have briefly described some of the changes you could apply:

\begin{itemize}
\item To run your implementation with/without a baseline you may change the ~use_baseline~ option to either ~true~ or ~false~.

\item You may also choose to train multiple seeds under the one python process through specifying more than one seed in the list ~seed~ (equally you may specify a single seed in this list if you want to run just one).

\item You may also wish to qualitatively observe the performance of your agent. To do so, you can record a single episode of the trained agent through changing the ~record~ option to ~true~ in the training configuration file for your run. Once your training run is complete you should see a video recording outputted in the ~results~ folder for the given run.
\end{itemize}

\textbf{Results \& Plotting}

Once training is complete you should observe the creation of the following folder ~src/results~ which contains the results of your training runs. In addition, you should note that some files based on model evaluation have been created in the submission folder. This contains the weights and scores for one of your training runs for each environment and baseline/no baseline configuration (this will only populate once one of your training runs achieves evaluation scores above a certain threshold). You will need to upload these weights with your submission to receive full credit for this question. To plot your results for certain seeds run:

\begin{lstlisting}
$ python run.py --plot_config_filename plot_config_filename
\end{lstlisting}

where plot\_config\_filename is the name of a yaml file in the ~src/config~ folder containing information on the seeds we wish to plot and for which environments. Please consult ~config/plot.yml~ for an example of the structure of this configuration file.

On the following page we provide sample outputs and an outline of the performance you can expect for the correct implementation.

\clearpage
\textbf{Expected Results}

We expect your plots to look similar to the plots which we have included:

\begin{figure}[H]
\centering
  \includegraphics[width=.45\linewidth]{images/CartPole-v1.png}
  \includegraphics[width=.45\linewidth]{images/InvertedPendulum-v4.png}
\end{figure}

\begin{figure}[H]
\centering
  \includegraphics[width=.45\linewidth]{images/HalfCheetah-v4.png}
\end{figure}

The following performance benchmarks need to be achieved by your implementation in order to receive full credit for this question:

\begin{itemize}
	\item CartPole-v1: agent should reach the maximum return of 200 with and without baseline (although it may not stay there)
	\item InvertedPendulum-v4: agent should reach the maximum return of 1000 with and without baseline (although it may not stay there)
	\item HalfCheetah-v4: agent should reach at least a score of 200 with and without baseline (in general it can achieve a much higher score e.g. 900)
\end{itemize}
\clearpage


\end{enumerate}